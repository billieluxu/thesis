
\documentclass[12pt]{report}
\usepackage[utf8]{inputenc}
\usepackage{graphicx}
\usepackage{listings}
\usepackage{lstautogobble}
\usepackage{amsmath} 
\graphicspath{ {images/} }
\usepackage{float}
\floatstyle{boxed} 
\restylefloat{figure}


\renewcommand{\|}{\;|\;}

\begin{document}

Our presentation of Tendermint consensus draws heavily on Nestmann (2003),
which models the non-Byzantine consensus protocol of Chandra and Touegg (1996).
We model only the consensus, rather than full ABC, but describe how to 
easily extend the model to ABC.

Let $Consensus := \prod_{i=1}^N Y_i $ represent the tendermint consensus protocol
over a set of $N$ validators, each executing one of a mutually exclusive set of processes, $Y_i$.
Internal state $s = \{r, p, v, l\}$ consists of a strictly increasing round, $r$,
a proposal $p$, containing the proposed block for this round;
a set of votes, $v$, containing all votes at all rounds;
and the lock information, $l$, which contains information about the currently locked block, if there is one (see below).
We denote by $v_r^1$ and $v_r^2$ the set of prevotes and precommits, respectively, at round $r$.
We define $proposer(r) = r \mod n$ to be the index of the proposer at round $r$.
We represent a peer at a particular point in the protocol as $Y_i^{r, p, v, l}$.
Processes $Y_i$ range over $PR_i$, $PV_i$, $PC_i$, $C_i$,
respectively abbreviating 
\emph{propose}, \emph{prevote}, \emph{precommit}, \emph{commit}.
We introduce additional sub-functions for $PV$ and $PC$ to capture the recursion,
denoted $PV1$, $PV2$, etc.

Peers are connected using broadcast channels for each message type,
namely $propose_i$, $prevote_i$, and $precommit_i$,
as well as a channel for broadcasting new transactions, $b_i$,
and one for deciding on, or committing, the next block, $d_i$.
Via an abuse of notation, a single send on some $x_i$ can be received by each process along
$x_i$.

We use only two message types: proposals and votes. 
Each contains a round number, block (hash), and signature, 
denoted $msg.round$, $msg.block$, $msg.sig$.
Note we can absorb the signature into the broadcast channel itself,
but we need it for use as evidence in the event of byzantine behaviour.


\begin{center}
	\begin{tabular}{l }
		\hline \\
		$Consensus := \prod_{i=1}^N [ PR_i^{0,\emptyset,\emptyset,\emptyset} \| D_i]$ \\\\

		\hline \\
		{$\!\begin{aligned}
		PR_i^{r,p,v,l} := 
			& \text{if } i=proposer(r) \text{ then } \\
				& \quad propose_i ! (prop) \| PV_i^{r,prop,v,l} \text{, where } prop = chooseProposal(p, l)\\
			& \text{ else if } l \neq \emptyset \text{ then}  \\
				& \quad PV_i^{r,l,v,l}  \\
			& \text{else} \\ 
				& \quad propose_{proposer(r)} ? (prop).PV_i^{r,prop,v,l} + susp_{proposer(r)}.PV_i^{r,nil,v,l} \\
		\end{aligned}$} \\\\

		\hline \\
		$PV_i^{r,p,v,l}:= prevote_i ! (r,p) \| (\nu \> c) ( \prod_{j=1}^n prevote_j ? (w) . c!(w)  \| PV1_i^{r,p,v,l}(c))$ \\\\

		\hline \\
		{$\!\begin{aligned}
		PV1_i^{r,p,v,l}(c) := 
			& \text{ if } max_{b}(|\left\{ w \in v_r^1 : w.block = b\right\}|) > \frac{2}{3} N \text{ then} \\
				& \quad PC_i^{r,b,v,b} \\
			& \text{else if }  | v_r^1 | > \frac{2}{3} N \text{ then} \\ 
				& \quad PC_i^{r,nil,v,l} \\ 
			& \text{else} \\
				& \quad c?(vote) . PV1_i^{r,p,vote::v,l}(c) \\
		\end{aligned}$} \\\\

		\hline \\
		$PC_i^{r,p,v,l}:= precommit_i ! (r,p) \| (\nu \> c) ( \prod_{j=1}^n precommit_j ? (w) . c!(w)  \| PC1_i^{r,p,v,l}(c))$ \\\\

		\hline \\
		{$\!\begin{aligned}
		PC1_i^{r,p,v,l}(c) := 
			& \text{ if } max_{b}(|\left\{ w \in v_r^2 : w.block = b\right\}|) > \frac{2}{3} N \text{ then} \\
				& \quad C_i^{r,b,v,l} \\
			& \text{else if }  | v_r^2 | > \frac{2}{3} N \text{ then} \\ 
				& \quad PR_i^{r+1,\emptyset,v,l} \\ 
			& \text{else} \\
				& \quad c?(vote) . PC1_i^{r,p,vote::v,l}(c) \\
		\end{aligned}$}


	\end{tabular}
\end{center}










\end{document}
