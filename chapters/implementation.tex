\chapter{Implementation}
\label{ch:implementation}

The reference implementation of Tendermint is written in Golang \ref{golang} and hosted at https://github.com/tendermint/tendermint.
Golang is a C-like language with a rich standard library, concurrency primitives for light-weight massively concurrent executions,
and a development environment optimized for simplicity and efficiency.

\section{Libraries}

The code uses a number of packages which are modular enough to be isolated as their own libraries.
These packages were written in the most part by Jae Kwon, with bug fixes, tests, and the occasional feature contributed by the author.
The most important of these packages are described in the following sub-sections.

\subsection{Binary Serialization}

Tendermint uses a simple binary serialization algorithm for packing memory objects into byte arrays and back.


\subsection{Cryptography}
\subsection{Merkle Hash Tree}
\subsection{P2P Networking}

The P2P networking used by Tendermint is described more fully in Chapter \ref{ch:networking}.

\subsection{RPC}

\section{Reactors}

NOTE: this should be moved to Chapter \ref{ch:networking}

The Tendermint node is composed of multiple concurrent reactors, 
each managing a state machine sending and receiving messages to peers over the network.
Reactors synchronize by locking shared datastructures, but the points of synchronization are kept to a minimum,
so each reactor runs mostly concurrently with the others.

\subsection{Mempool}

The mempool reactor manages the mempool, 
which caches transactions before they are packed in blocks and committed.
The mempool state machine uses a subset of the application's state machine to check the validity of transactions ...

\subsection{Consensus}

The consensus reactor manages the consensus state machine, which handles proposals, voting, locking, 
and the actually committing of blocks.
The state machine is managed using a few persistent go-routines, 
which order received messages and enable them to be played back deterministically to debug the state.
These go-routines include the readLoop, for reading off the queue of receivied messages, 
and the timeoutLoop, for registering and triggering timeout events. 

Transitions in the consensus state machine are made either when a complete proposal and block are received,
or when more than two-thirds of either pre-votes or pre-commits have been received at a given round.
Transitions result in the broadcast of proposals, block data, or votes, which are queued on the internalReqQueue,
and processed by the readLoop in serial with messages received from peers.
This puts internal messages and peer messages on equal footing as far as being inputs to the consensus state machine, 
but allows internal messages to be processed faster, as they don't sit in the same queue as those from peers.

\subsection{Blockchain}

The blockchain reactor syncs the blockchain using a much faster technique than the consensus reactor.
Namely, validators request blocks of incrementing height until none of their peers have blocks of any higher height.
Blocks are collected in a blockpool and synced to the blockchain by a worder routine that periodically takes blocks from the pool
and validates them against the current chain.

Once the blockchain reactor finishes syncing up, it turns on the consensus reactor to take over.

\section{Other Considerations}
