\chapter{Client Considerations}
\label{ch:clients}

This chapter reviews some considerations pertaining to clients that interact with an application hosted on Tendermint.

\section{Discovery}

Network discovery occurs simply by dialing some set of seed nodes over TCP.
The p2p network uses authenticated encryption, 
but the public keys of the validators must be verified somehow out of band, 
that is, via an alternative medium not within the purview of the protocol.
Indeed, in these systems, the genesis state itself must be communicated out of band, 
and ideally is the only thing that must be communicated, 
as it should also contain the public keys used by validators for authenticated encryption, 
which are different than those used for signing votes in consensus.

For validator sets that may change over time, it is useful to register all validators via DNS, 
and to register new validators before they actually become validators, and remove them after they are removed as validators.
Alternatively, validator locations can be registered in another fault-tolerant distributed data store, 
including possibly another Tendermint cluster itself.

\section{Broadcasting Transactions}

As a generalized application platform, Tendermint provides only a simple interface to clients for broadcasting transactions.
The general paradigm is that a client connects to a Tendermint consensus network through a proxy, which is either run locally on its machine,
or hosted by some other provider. The proxy functions as a non-validator node on the network, 
which means it keeps up with the consensus and processes transactions, but does not sign votes.
The proxy enables client transactions to be quickly broadcast to the whole network via the gossip layer.

A node need only connect to one other node on the network to broadcast transactions, but by default will connect to many,
minimizing the chances that the transaction will not be received.
Transactions are passed into the mempool, 
and gossiped through the mempool reactor to be cached in the mempool of all nodes, 
so that eventually one of them will include it in a block. 

Note that the transaction does not execute against the state until it gets into a block,
so the client does not get a result back right away, other than confirmation that it was accepted into the mempool and broadcast to other peers.
Clients should register with the proxy to receive the result as a push notification when it is computed during the commit of a block.

It is not essential that a client connect to the current proposer, 
as eventually any validator which has the transaction in its mempool may propose it.
However, preferential broadcasting to the next proposer in line may lead to lower latency for the transaction
in certain cases where the network is under high load. Otherwise, the transaction should be quickly gossiped to every validator.

\section{Mempool}

The mempool is responsible for caching transactions in memory before they are included in blocks.
Its behaviour is subtle, and forms a number of challenges for the overall system architecture.
First and foremost, caching arbitrary numbers of transactions in the mempool is a direct denial of service attack
that could trivially cripple the network. Most blockchains solve this problem using their native currency,
and permitting only transactions which spend a certain fee to reside in the mempool.

In a more generalized system, like Tendermint, where there is not necessarily a currency to pay fees with,
the system must establish stricter filtering rules and rely on more intelligent clients to resubmit transactions that are dropped.
The situation is even more subtle, however, because the rule set for filtering transactions in the mempool must be a function of the application itself.
Hence the \emph{CheckTx} message of TMSP,
which the mempool can use to run a transaction against a transient state of the application to determine if it should be kept around or dropped.

Handling the transient state is non-trivial, and is something left to the application developer, 
though examples are provided in the many example applications. 
In any case, clients must monitor the state of the mempool (i.e.~the unconfirmed transactions) to determine if they need to rebroadcast their transactions,
which may occur in highly concurrent settings where the validity of one transaction depends on having processed another.

\section{Semantics}

Tendermint's core consensus algorithm provides only \emph{at-least-once semantics}, 
which is to say the system is subject to replay attacks, 
where the same transaction can be committed many times.
However, many users and applications expect stronger guarantees from a database system.
The flexibility of the Tendermint system leaves the strictness of these semantics up to the application developer.
By utilizing the \emph{CheckTx} message, and by adequately managing state in the application, 
application developers can provide the database semantics that suit them and their users' needs.
For instance, as discussed in Chapter \ref{ch:apps}, 
using an account based system with sequence numbers mitigates replay attacks,
and changes the semantics from \emph{at-least-once} to \emph{exactly-once}.

\section{Reads} 

Clients issue read requests to the same proxy node they use for broadcasting transactions (writes).
The proxy is always available for reads, even if the network halts.
However, in the event of a partition, the proxy may be partitioned from the rest of the network, which continues making blocks.
In that case, reads from the proxy might be stale.

To avoid stale reads, the read request can be sent as a transaction, presuming the application permits such queries.
By using transactions, reads are guaranteed to return the latest committed state, i.e.~when the read transaction is committed in the next block.
This is of course much more expensive than simply querying the proxy for the state.
It is possible to use heuristics to determine if a read will be stale,
such as if the proxy is well-connected to its peers and is making blocks, 
or if it's stuck in a round with votes from one-third or more of validators,
but there is no substitute for performing an actual transaction.

\section{Light Client Proofs}

One of the major innovations of blockchains over traditional databases is their deliberate use of Merkle hash trees to enable the production
of compact proofs of system substates, so called light-client proofs.
A light client proof is a path through a Merkle tree that allows a client to verify that some key-value pair is in the Merkle tree with a given root hash.
The state's Merkle root hash is included in the block header, such that it is sufficient for a client to have only the latest header to verify any component of the state.
Of course, to know that the header itself is valid, they must have either validated the whole chain, 
or kept up-to-date with validator set changes only and rely on economic guarantees that the state transitions were correct.


