\chapter{Conclusion}

Byzantine Fault Tolerant consensus provides a rich basis upon which to build services 
that do not depend on centralized, trusted parties, and which may be adopted by society
to manage critical components of socioeconomic infrastructure.
Tendermint, as presented in this thesis, was designed to meet the needs of such systems,
and to do so in a way that is understandably secure and easily high performance,
and which allows arbitrary systems to have transactions ordered by the consensus protocol,
with minimal fuss.

Careful considerations are necessary when deploying a distributed consensus system,
especially one without an agreed upon central authority to mediate potential disputes and reset the system in the event of a crisis.
Tendermint seeks to address such problems using explicit governance modules and accountability guarantees,
enabling integration of Tendermint deployments into modern legal and economic infrastructure.

There is still considerable work to do. This includes formal verification of the algorithm's guarantees, 
performance optimizations, and architectural changes to enable the system to increase capacity with the addition of machines.
And of course, many, many TMSP applications remain to be built.

We hope that this thesis better illuminates some of the problems in distributed consensus and blockchain architecture,
and inspires others to build something better.
