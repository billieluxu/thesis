\chapter{Governance}

So far, this thesis has reviewed the basic elements of the Tendermint consensus protocol and application environment.
Critical elements of operating the system in the real world, such as managing validator set changes
and recovering from crisis, have not yet been discussed. 

This chapter proposes an approach to these problems that formalizes the role of governance in a consensus system.
As validator sets come to encompass more decentralized sets of agents, competent governance systems 
for maintaining the network will be increasingly paramount to success.

\section{Governmint}

The basic functionality of governance is to filter proposals for action, typically through a form of voting.
The most basic implementation of governance as software is a module that enables users to make proposals,
vote on them, and tally the votes. Most proposals may be related to things outside the reach of the software.
Others might be within the reach of the software, and hence potentially executable.

To enable certain actions in Tendermint, such as changing the validator set or upgrading the software,
a governance module has been implemented, called Governmint.
Governmint is a minimum viable governance application with support for multiple groups of entities,
each of which can vote internally on proposals, some of which may result in programmatic execution of actions,
like changing the validator set, or upgrading governmint itself (for instance to add new proposal types or other voting mechanisms).

\section{Validator Set Changes}

\section{Software Upgrades}

\section{Crisis Recovery}
