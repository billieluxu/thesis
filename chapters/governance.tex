\chapter{Governance}
\label{ch:governance}

So far, this thesis has reviewed the basic elements of the Tendermint consensus protocol and application environment.
Critical elements of operating the system in the real world, such as managing validator set changes
and recovering from a crisis, have not yet been discussed. 

This chapter proposes an approach to these problems that formalizes the role of governance in a consensus system.
As validator sets come to encompass more decentralized sets of agents, competent governance systems 
for maintaining the network will be increasingly paramount to the network's success.

\section{Governmint}

The basic functionality of governance is to filter proposals for action, typically through a form of voting.
The most basic implementation of governance as software is a module that enables users to make proposals,
vote on them, and tally the votes. 
Proposals may be programmatic, in which case they may execute automatically following a successful vote,
or they may be non-programmatic, in which case their execution is a manual exercise.

To enable certain actions in Tendermint, such as changing the validator set or upgrading the software,
a governance module has been implemented, called Governmint.
Governmint is a minimum viable governance application with support for multiple groups of entities,
each of which can vote internally on proposals, some of which may result in programmatic execution of actions,
like changing the validator set, or upgrading Governmint itself (for instance to add new proposal types or other voting mechanisms).

The system utilizes digital signatures to authenticate voters, 
and may use a variety of possible voting schemes.
Of particular interest are quadratic voting schemes,
where the cost to vote is quadratic in the weight of the vote,
which have been shown to have a superior ability to satisfy voter preferences \cite{posner2013quadratic}.

\section{Validator Set Changes}

Validator set changes are a critical component of real world consensus algorithms that many previous approaches have failed to specify 
or have been left as a black art. 
Raft took pains to expound a sound protocol for validator set changes, which required the change pass through consensus, 
using a new message type.
Tendermint takes a similar approach, though it is standardized through the TMSP interface using the \emph{EndBlock} message,
which is run after all the \emph{AppendTx} messages, but before \emph{Commit}.
If a transaction, or set of transactions, is included in a block with the intended effect of updating the validator set,
the application can return a list of validators to update by specifying their public key and new voting power 
in response to the \emph{EndBlock} message.
Validators can be removed by setting their voting power to zero.
This provides a generic means for applications to update the validator set without having to specify transaction types.

If the block at height $H$ returns an updated validator set, 
then the block at height $H+1$ will reflect the update.
Note, however, that the \emph{LastCommit} in block $H+1$
must utilize the validator set as it was at $H$, 
since it may contain signatures from a validator that was removed.

Changes to voting power are applied for $H+1$ such that the next proposer 
is affected by the update. 
In particular, the validator that otherwise should have been the next proposer may be removed.
The round robin algorithm should handle this gracefully, simply moving on to the next proposer in line.
Since the same block is replicated on at least two-thirds of validators, 
and the round robin is deterministic,
they will all make the same update and expect the same next proposer.

\section{Punishing Byzantine Validators}

One of the salient points of Bitcoin's design is its incentive structure, 
in so far as the goal of the protocol was to incentivize validators to behave correctly
by rewarding them. While this makes sense in the context of Bitcoin's consensus protocol,
a superior incentive may be to provide strong dis-incentives, such that validators
have real \emph{skin-in-the-game} \cite{taleb2014skin}, rather than a soft opportunity cost.

Disincentives can be achieved in Tendermint using an approach first proposed by Vitalik Buterin \cite{slasher} as a so-called Proof-of-Stake protocol.
In essence, validators must make a security deposit (``they must bond some stake'')
in order to participate in consensus.
In the event that they are found to double-sign proposals or votes, 
other validators can publish evidence of the transgression in the form of a transaction, 
which the application state can use to change the validator set by removing the transgressor, burning its deposit.
This has the effect of associating an explicit economic cost with Byzantine behaviour, 
and enables one to estimate the cost of violating safety by bribing a third or more of the validators to be Byzantine.

Note that a consensus protocol may specify more behaviours to be punished than just double signing.
In particular, we are interested in punishing any strong signalling behaviour which is unjustified - typically, any reported change in state that is not based on the reported state of others.
For instance, in a version of Tendermint where all pre-commits 
must come with the polka that justifies them,
validators may be punished for broadcasting unjustified pre-commits.
Note, however, that we cannot just punish for any unexpected behaviour - 
for instance, a validator proposing when it is not their round to propose
may be a basis for optimizations which pre-empt asynchrony or crashed nodes.

In fact, a generalization of Tendermint along these two lines, 
of 1) looser forms of justification and 2) allowing validators to propose before their term,
gives rise to a family of protocols similar in nature to that proposed by Vlad Zamfir,
under the guise Casper, as the consensus mechanism for a future version of ethereum \cite{casper}.
A more formal account of the relationship between the protocols, 
and of the characteristics of anti-Byzantine justifications, remains for future work.

\section{Software Upgrades}

Governmint can also be used as a natural means for negotiating software upgrades on a possibly decentralized network.
Software upgrades on the public Internet are a notoriously challenging operation,
requiring careful planning to maintain backwards compatibility for users that don't upgrade right away,
and to not upset loyal users of the software by introducing bugs, removing features, adding complexity, or,
perhaps worst of all, updating automatically without permission.

The challenge of upgrading a decentralized consensus system is made especially apparent with Bitcoin.
While Ethereum has already managed a successful, non-backwards-compatible upgrade, 
due to its strong leadership and unified community,
Bitcoin has been unable to make some needed upgrades,
despite a plethora of software engineering ills,
on account of a viciously divided community and a lack of strong leadership.

Upgrades to blockchains are typically differentiated as being \emph{soft forks} or \emph{hard forks},
on account of the scope of the changes.
Soft forks are meant to be backwards compatible, and to use degrees of freedom in the protocol that may be ignored
by users who have not upgraded, but which provide new features to users which do.
Hard forks, on the other hand, are non-backwards compatible upgrades that,
in Bitcoin's case, may cause violations of safety, 
and in Tendermint's case, cause the system to halt.

To cope, developers of the Bitcoin software have rolled out a series of soft forks for which validators can vote by signalling in new blocks. 
Once a certain threshold of validators are signalling for the update,
it automatically takes effect across the network, at least for users with a version of the software supporting the update.
The utility of the Bitcoin system has grown tremendously on account of these softforks, 
and is expected to continue to do so on account of upcoming ones.
Interestingly, the failure of the community to successfully hard fork the software has
on the one hand raised concerns about the long term stability of the system,
and on the other triggered excitement and inspiration about the system's resilience to corrupt governance - its ungovernability.

There are many reasons to take the latter stance, 
given the overwhelming government corruption apparent in the world today.
Still, cryptography and distributed consensus provide a new set of tools that enables a degree
of transparency and accountability otherwise not imaginable in the paper-pen-handshake world of modern governments,
nor even the digital world of the traditional web, which suffers tremendously from sufficiently robust authentication systems.

In a system using Governmint, developers would be identifiable entities on the blockchain,
and may submit proposals for software upgrades. 
The mechanism is quite similar to that of a Pull Request on Github, 
only it is integrated into a live running system,
and the agreement passes through the consensus protocol.
Clients should be written with configurable update parameters, 
so they can specify whether to update automatically or to require that they are notified first.

Of course, any software upgrade which is not thoroughly vetted could pose a danger to the system,
and a conservative approach to upgrades should be taken in general.

\section{Crisis Recovery}

In the event of a crisis, such as a fork in the transaction log,
or the system coming to a halt, 
a traditional consensus system provides little or no guarantees,
and typically requires manual intervention.

Tendermint assures that those responsible for violating safety can be identified,
such that any client who can access at least one honest validator 
can discern with cryptographic certainty who the dishonest validators are,
and thereby chose to follow the honest validators onto a new chain with a validator set excluding those who were Byzantine.

For instance, suppose a third or more validators violate locking rules,
causing two blocks to be committed at height $H$.
The honest validators can determine who double-signed by gossipping all the votes.
At this point, they cannot use the consensus protocol, because the basic fault assumptions have been violated.
One approach is for each validator to wait until they have all 
Note that being able to at this point accumulate all votes for $H$ 
implies strong assumptions about network connectivity and availability during the crisis,
which, if it cannot be provided by the p2p network, may require validators use alternative means, 
such as social media and high availability services, to communicate evidence.
A new blockchain can be started by the full set of remaining honest nodes, 
once at least two-thirds of them have gathered all the evidence.

Alternatively, modifying the Tendermint protocol so that pre-commits require polka
would ensure that those responsible for the fork could be punished immediately,
and would not require an additional publishing period. 
This modification remains for future work.

More complex uses of Governmint are possible for accommodating various particularities of crisis,
such as permanent crash failures and the compromise of private keys.
However, such approaches must be carefully thought out, 
as they may undermine the safety guarantees of the underlying protocol.
We leave investigation of these methods to future work, 
but note the importance of the socio-economic context in which a blockchain is embedded, in terms of understanding its ability to recover from crisis.

Regardless of how crisis recovery proceeds, its success depends on integration with clients.
If clients do not accept the new blockchain, the service is effectively offline.
Thus, clients must be aware of the rules used by the particular blockchain to recover.
In the cases of safety violation described above, they must also gather the evidence,
determine which validators to remove, and compute the new state with the remaining validators.
In the case of the liveness violation, they must keep up with Governmint.

\section{Conclusion}

Governance is a critical element of a distributed consensus system, 
though competent governance systems remain poorly understood.
Tendermint provides governance as a TMSP module called Governmint,
which aims to facilitate increased experimentation in software-based governance for distributed systems.

